\section{Methods}
\label{sec:methods}
\sksglenoid currently implements two 3D methods; the two-plane method described by Ganapathi et al. \cite{PMID:20933439} and the 3D corrected 
Friedman method described by Budge et al. \cite{BUDGE2011577}. \sksglenoid also implements two
2D methods; Friedman's method \cite{PMID:1522089} and the vault method described by Matsumura et al. \cite{PMID:24618285}. Each implementation can be accessed by via a command line application which takes as 
input a file describing the anatomical position of the required landmark points. 

\sksglenoid is built on top of the \sksurgery \cite{PMID:32436132} libraries which enabled 
rapid development and future deployment into a clinically useable application.  
\sksglenoid currently requires these landmarks to be manually defined. We performed segmentation 
and landmark annotation using 3DSlicer \cite{Kikinis2014} on 10 patients and processed
the resulting 
segmentation using \sksglenoidns.

Statistical analysis was performed using GraphPad Prism software version 8.0 for
Mac \footnote{GraphPad Software, San Diego, CA, USA}. The mean and standard deviation of 
each method was calculated and compared. Pearson’s correlation coefficient was 
determined between the commercial software and each method. 
A repeated measures ANOVA was performed to determine any significant differences
in version measurements between the methods. Significance level for all analyses was set at 0.05.
