\section{Methods}
\label{sec:methods}
\sksglenoid currently implements the two plane method described by Ganapathi et al. \cite{PMID:20933439}, Friedman's method \cite{PMID:1522089}, the 
corrected Freidman method \cite{BUDGE2011577}, and the vault method described by Matsumura et al. \cite{PMID:24618285}. Each implementation can be accessed by via a command line application which takes as 
input a file describing the anatomical position of the required landmark points. 
\sksglenoid is built on top of the \sksurgery \cite{PMID:32436132} libraries which enabled 
rapid development and future deployment into a clinically useable application.  
\sksglenoid currently requires these landmarks to be manually defined. We performed segmentation 
and landmark annotation using 3DSlicer \cite{Kikinis2014} on 10 patients and processed the resulting 
segmentation using \sksglenoid.
