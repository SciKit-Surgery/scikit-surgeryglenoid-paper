\section{Introduction}
\label{sec:introduction}
Correct understanding of the geometry of the glenoid (the socket of the shoulder joint) is
key to successful planning of shoulder replacement surgery. This surgery typically involves
placing an implant in the shoulder joint to restore joint function. The most relevant
geometry is the glenoid version, which is the angular orientation of the glenoid surface
relative to the long axis of the scapula in the horizontal (axial) plane. 
However, measuring the glenoid version is not
straightforward and there are multiple measurement methods in the literature and used in
commercial planning software. 

Glenoid version can be computed from {2D} radiographs or from {3D} {CT} scans.
The use of radiographs for glenoid measurement has been shown to be less 
reliable that {CT} based methods \cite{nyffeler2003measurement}, so 
most modern approaches use {CT} scans for glenoid version measurement. 
Methods using {CT} scans can be divided into {2D} methods that 
use a single axial slice to estimate glenoid version and {3D} 
methods that use landmarks points in multiple slices. {2D} methods 
have been shown to be more susceptible to positional variance \cite{bryce2010two}, 
however there is not as yet an agreed single method for glenoid version measurement.
There are many papers comparing the effect of use 
different methods for measuring glenoid version and proposing their own 
methods, \cite{PMID:33330245, PMID:32010231, PMID:29298261, PMID:33554174}. 
Different methods are also implemented by implant vendors and commercial 
software suppliers \cite{blueprint, exactech, djosurgical} however the exact methods
used by in each case are not published. 

There is a need therefore for reliable open source implementations of the various 
methods for measuring glenoid version to enable further research comparing the methods. 
We have developed \sksglenoid to meet this need and herein present early results of \sksglenoidns's 
use on retrospectively gathered data.
