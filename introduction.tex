\section{Introduction}
\label{sec:introduction}
Correct understanding of the geometry of the glenoid (the socket of the shoulder joint) is
key to successful planning of shoulder replacement surgery. This surgery typically involves
placing an implant in the shoulder joint to restore joint function. The most relevant
geometry is the glenoid version, which is the angular orientation of the glenoid surface
relative to the long axis of the scapula in the horizontal (axial) plane. 
However, measuring the glenoid version is not
straightforward and there are multiple measurement methods in the literature and used in
commercial planning software. Several papers look at the effect of using 
different methods for measuring glenoid version and propose their own 
methods, \cite{PMID:33330245, PMID:32010231, PMID:29298261, PMID:33554174}. 
Different approaches are also implemented by implant vendors and commercial 
software suppliers \cite{blueprint, exactech, djosurgical} however these methods can 
be somewhat opaque.

There is a need therefore for reliable open source implementations of the various 
methods for measuring glenoid version to enable further research comparing the methods. 
We have developed \sksglenoid to meet this need and herein present early results of its 
use on retrospectively gathered data.
