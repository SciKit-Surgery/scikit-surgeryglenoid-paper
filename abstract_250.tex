%https://spie.org/MI22/conferencedetails/image-guided-procedures?enableBackToBrowse=true

Correct understanding of the geometry of the glenoid (the dish of the shoulder joint) is 
key to successful planning of shoulder arthroplasty. This surgery typically involves 
placing an implant in the shoulder joint to restore joint function. The most relevant 
geometry is the glenoid version, which is the angular orientation of the glenoid surface 
relative to the long axis of the scapula. However, measuring the glenoid version is not 
straightforward and there are multiple measurement methods in the literature and used in 
commercial planning software. 

In this paper we introduce SciKit-SurgeryGlenoid, an open source toolkit for the measurement
of Glenoid version. SciKit-SurgeryGlenoid contains implementations of the 4(5?) most 
frequently used glenoid version measurement algorithms enabling easy and unbiased comparison of 
the different techniques. We present the results of using the software on 10(?) sets of
CT scans taken from patients who had undergone shoulder implant surgery.    

Here we describe the software and present results based on manual segmentation of 10 
patients. We further compare these results with those obtained from a commercial 
implant planning software. 

SciKit-SurgeryGlenoid currently requires manual segmentation of the relevant anatomical 
features for each method. Future work will look at automating the segmentation process
 to build an automatic and repeatable 
pipeline from {CT} or radiograph to quantitative glenoid version measurement.
