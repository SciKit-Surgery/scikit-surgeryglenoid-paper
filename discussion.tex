\section{Discussion}
\label{sec:discussion}
There are several methods that have been proven to be accurate in preoperative measurement of 
the glenoid version. Specifically, 3D methods have become the standard as they
provide a higher accuracy accounting for the positional errors during image acquisition
(Budge et al. \cite{BUDGE2011577}, Moineau et al. \cite{PMID:22964089}). 
Testing the most common 2D and 3D methods using the 
\sksglenoid toolkit allowed for an evaluation of its effectiveness in comparing these methods. 
The early results presented are consistent with 
previously reported results (Matsumura et al. \cite{PMID:24618285}, 
Budge et al. \cite{BUDGE2011577}, Ganapathi et al. \cite{PMID:20933439}).

While the mean version did not show any significant difference between most methods, 
this could be due to the small sample size used in this case. However, it is notable
that the 3D methods reveal slightly lower version means and lower standard deviations
which could prove significant when more scans are analysed with additional observers.

From the Pearson correlation coefficient, significant correlation between the commercial 
software and 3 out of the 4 methods was seen. The vault method showing little correlation
with the commercial software could be due to its much higher mean version value.
The vault method tends to overestimate the glenoid version as has been previously
reported by several studies (Cunningham et al. \cite{PMID:29778592},
Matsumura et al. \cite{PMID:24618285}). 
The correlation
tests prove however that there is good agreement between \sksglenoid and the
commercial software already in use, indicating accuracy and credibility of this toolkit.
While this can good indicator of reliability, further measurements using this software
by different observers would be needed to be able to test inter and intra observer reliability.
However, \sksglenoid proves to be promising in providing an unbiased way of
comparing the many different methods available to measure glenoid version.

Limitations of this initial testing of \sksglenoid include the small sample size used.
A study 
with a wider range of CT scans could reveal better understanding of the software’s reliability. 
Additionally, as points for each method were selected manually, there are some inaccuracies that 
arise which could be better understood with repeated measurements and multiple observers.

\section{Conclusion}
\sksglenoid provides a useful resource for shoulder arthroplasty. Future work could look at either 
automating the segmentation process using state of the art registration algorithms \cite{Fu2020} to create a fully automatic pipeline, or at integrating the library with 3DSlicer to create a 
``slicelet" based application, similar to our previous work \cite{PMID:33937966} in skull base 
navigation.

